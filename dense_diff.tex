% !TEX TS-program = xelatex

\documentclass{article}
%\documentclass[landscape]{article}
\def\image{../../../../../Preamble/hselogo.png}

\input{../../../../../Preamble/lepreamble.tex}
\usetikzlibrary{patterns}
\renewcommand{\arraystretch}{1.185}

\pagestyle{fancy} 
\fancyhead{} 
\fancyhead[C]{\large }
\fancyhead[L]{\large \today} 
\fancyhead[R]{\large Плотность и дифференциалы}
\fancyfoot{} 
\fancyfoot[C]{\large \thepage}

\begin{document}
\large
\begin{btitleframe}{\Large Плотность и дифференциалы}
\flushleft{\textbf{Составители:} Александр Югай, Егор Фадеев, Анна Казачкова, Александр Ганибаев} \vspace{-0.6em}
\flushleft{\textbf{Группа:} БЭК181} \vspace{-0.6em}
\flushleft{\textit {\today}}
\vspace{1em}
\end{btitleframe}

\problem{
Точки равномерно распределены на множестве:

\[D=\{(x,\ y):\ x^2+y^2\le R^2\}\]

Найти функции плотности для абсциссы $f_X(x)$ и ординаты $f_{Y}(y)$ точки.
}

\problem{
Точки равномерно распределены на области, ограниченной прямыми $y=1-x,\ x=0,\ y=0$. Найти функции плотности для абсциссы $f_X(x)$ и ординаты $f_{Y}(y)$ точки.
}

\problem{
Точки равномерно распределены на области, ограниченной кривыми $\ln{x+1}, x=1, y=0$. Найти функции плотности для абсциссы $f_X(x)$ и ординаты $f_{Y}(y)$ точки.
}

\problem{
Заданы множества:

\begin{align}
    D_1 = \{(x,y): x^2+y^2 \le 4\} && D_2 = \{(x,y): (x-1)^2+y^2 \le 3\}
\end{align}

Точки равномерно распределены в множестве $A = D_1 \setminus D_2$. Найти функции плотности для абсциссы $f_X(x)$ и ординаты $f_{Y}(y)$ точки.
}

\problem{
Найти функции плотности $x$ и $y$ на фигуре, ограниченной кривыми $x = 0$, $y = e^x$ и $y = x^2$.
}

\problem{
Найти функцию плотности $x$ на фигуре, ограниченной кривыми $y=x^5$, $y=0$, $x=2$.
}

\problem{
Найти функцию плотности $x$ на фигуре, ограниченной прямыми $y = 0$, $y=x$ и $y =-x+2$.
}
%%%%%%%%%%%%%

\problem{
    Точки равномерно распределены на области, ограниченной кривыми $y=-(x-1)^2+1,\ y=0$. Найти функции плотности для абсциссы $f_X(x)$ и ординаты $f_{Y}(y)$ точки.
}

\problem{
    Две точки равномерно распределены на окружности с центром в начале координат и радиусом $R$. Найти плотность распределения расстояния между двумя точками.
}

\problem{
    Выведите функцию плотности $k$-й порядковой статистики из выборки равномерного распределения.
}

\problem{
    Вывести совместную функцию плотности для $k$-й и $m$-й порядковых статистик из выборки равномерного распределения.
}

\problem{
    На множество $D = \{(x\cm y)∶ ~~ x^2 + y^2 \leqslant R^2\}$ бросают n точек. Найти функцию плотности $X_k$ и $Y_k$, где $X_1\leqslant ... \leqslant X_k\leqslant ... \leqslant X_n$, $Y_1\leqslant ... \leqslant Y_k\leqslant ... \leqslant Y_n$.
}

\problem{
    $n$ точек равномерно упали на область, ограниченную прямыми $y=1–x$, $x = 0$, $y = 0$. Найти функции плотности для абсциссы $f_{X_k}(x)$ и ординаты $f_{X_k}(y)$ точки, где $X_k$ -- $k$-я порядковая статистика.
}

\problem{
    $n$ точек равномерно упали на область, ограниченную прямыми $ln(x + 1)$ , $x = 1$, $y=0$

    Найдите функции плотности для абсциссы $f_{X_k}(x)$ и ординаты $f_{X_k}(y)$ точки, где $X_k$ -- $k$-я порядковая статистика.
}
\end{document}
